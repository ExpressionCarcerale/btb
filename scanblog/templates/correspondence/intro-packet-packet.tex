

\documentclass[12pt]{article}
\usepackage{helvet}
\renewcommand{\familydefault}{\sfdefault}
\usepackage{graphicx}
\usepackage{textcomp}
\usepackage{url}
\usepackage{wrapfig}
\usepackage{titlesec}
\usepackage[letterpaper,left=1in,right=1in,top=0.8in,bottom=0.8in]{geometry}

\usepackage{fancyhdr}
\pagestyle{fancy}
\fancyhead[C]{\emph{Introductory Packet}}
\renewcommand{\headrulewidth}{0pt}

\setlength\parindent{0in}
\setlength\parskip{10pt}
\setlength\headsep{0in}
\titlespacing*{\section}{0pt}{0em}{-0.1em}
\titleformat*{\section}{\normalfont\bfseries}

\begin{document}

\centerline{\includegraphics[type=png,ext=.png,read=.png,width=5in]{{MEDIA_ROOT}}/intro/logo}}


{\bf \emph{Between the Bars}} is a blogging platform for anyone in America who is currently behind bars.  We'd like to introduce you to this free service, which is supported by the MIT Center for Civic Media.

Before you start, we want to make sure that you're aware of exactly what blogging is and how our service can be valuable to you -- but also, what the limitations of our service are.  We publish your letters (your own writing, artwork, or photos) that you send to us on the Internet, so that others can see them.  Please read this introductory packet and ask us any questions you have before sending any blog posts.



\section*{What is blogging?}
``Blogging'' is a way to publish letters, stories, poetry, or artwork on the Internet.  Things published on a ``blog'' (pronounced like ``dog'') can be read by anyone anywhere who has a computer and an Internet connection.  For example, your post could be read by someone in Japan, in Europe, or in the same town.


\begin{wrapfigure}{r}{2in}
\vspace{-20pt}
\includegraphics[type=png,ext=.png,read=.png,width=2in]{{MEDIA_ROOT}}/intro/blogroll}
\end{wrapfigure}


Blogging is different from pen-pal or personals services.  Those services only offer you a place to put a brief description of yourself and a photo.  By contrast, a blog is a place you can continue to send letters, news updates, new drawings, new stories, and more, which form a timeline or journal.  While you can have a ``profile'' on your blog site (a short introduction about yourself), it's not the main attraction: people come to your blog to see your writing or artwork.

You can have as many posts as you like, and they never need to be removed -- they just get pushed onto back pages.  On the front page of your blog, a small section (usually the first paragraph or two) of your post appears at the top, and small sections of each previous post appear beneath that.  Internet visitors can click ``read more'' at the bottom of the snippet to read the rest of your post.  You can think of it as something like the front page of a newspaper, which has ``continued on page A4'' at the bottom of the first few paragraphs.  Unlike a newspaper, your blog has essentially unlimited space.  The newest blog posts are always shown first.



\section*{What does it cost?}
Nothing, aside from the postage for your letters.  If you wish for us to return a photo or blog post, please include a stamped return envelope (and tell us what you want us to return!).  But otherwise, no return envelopes are necessary.  Please note: we don't keep the hard copies of posts for very long, so if you want something back, please tell us \emph{at the time you send it to us}, or it might have already been recycled.



\section*{How can others find out about my blog?}
Everyone with an Internet connection can freely access your blog.  Internet visitors to your blog can read your posts, and leave comments on your posts.  When you sign up, we'll send you your own Internet address (or ``URL'') which you can share with friends or family so they can access your blog.


\begin{wrapfigure}[5]{R}{1.5in}
\vspace{-30pt}
\includegraphics[type=png,ext=.png,read=.png,width=1.5in]{{MEDIA_ROOT}}/intro/profile}
\vspace{-40pt}
\end{wrapfigure}


\section*{Can I set up a profile page, like a pen pal site?}
Yes! To set up a profile page, craft a single beautiful page (including an attached photo, if you like) and include a note in your letter explaining that you want this page to be your profile.  Profiles can only be a single page, up to 8.5" x 11".  You might wish to include your name, your address, a photo, or a brief biography.



\section*{How does my letter become a blog post?}


\begin{wrapfigure}{L}{1.5in}
\vspace{-20pt}
\includegraphics[type=png,ext=.png,read=.png,width=1.5in]{{MEDIA_ROOT}}/intro/scanning}
\vspace{-50pt}
\end{wrapfigure}


When you send us a letter, volunteers and staff use a ``scanner'' to turn the paper document into images on a computer.  These images are then shown on your blog site, and they look just like the letter that you sent us.  Visitors to your site can also transcribe your writing (turning it into computer typing), which makes them more accessible, but people always see your own writing first.



\section*{Can I get printouts of my blog posts?}
We will send you a printout of your first blog post so you can see how it looks.  But from that point forward, we won't provide printouts, due to the cost and time required.  If for some reason we are unable to publish something you have sent us, we will always tell you and return it to you.



\section*{How do comments work?}
When people on the Internet visit your site, they can leave comments, or replies, on your blog posts.  These comments are visible to anyone, including other visitors to your blog.  Periodically (once every 1 to 2 weeks) we will print out any comments you have received, and mail them to you.  You may not get comments (not every blog does), but your posts are still there and being read by other people whether you get comments or not.  You can also reply to comments people send.



\section*{Can I name my blog?  Can I set up more than one blog, or blog with other people?}
If you'd like to name your blog (for example, to give it a theme), please include a note in a  letter to us telling us what you would like your blog to be named.  No profanity in blog names, please.  Currently, we can only set up one blog site for you, and we only support blogs with a single author.



\pagebreak
\centerline{ {\bf \underline{ MOST IMPORTANT TO KNOW }}}



\section*{What can I send?}
You can send us your stories, pieces of artwork, poems, or essays.  You're welcome to include photos, drawings, etc. in your posts.  But REMEMBER that anything you put on your blog will be visible to \emph{anyone}, including parole officers, the warden, CO's, and your relatives.  And everything well be published \emph{\underline{in your real name}}.



\section*{What shouldn't I send?}
We ask you to {\bf \underline{never}} send us anything illegal, inflammatory, or which might get you in trouble or affect your chances for parole.  Here are some guidelines for what we will not publish:

\begin{itemize}

    \item{ Never send anything that will get you in trouble. }
    \item{ Never send personal medical records, criminal records, court documents, mental health evaluations, certifications, or anything of that sort.\\ 
        {\bf \emph{\underline{Instead:}}} Send a letter describing your experience in your own words.
    }
    \item{ Never send in documents written by other people (without their explicit permission), and never include documents with signatures from third parties.\\
        {\bf \emph{\underline{Instead:}}} Get explicit permission from all third party authors before sending in documents written by them, and remove any signatures from people other than you.
    }
    \item{ Don't send anything that includes the full names of other people in prison, including both employees and incarcerated people.  It's great to tell stories about your experiences in prison, but please respect the privacy of others by using only first names or fake names when you do so.\\
        {\bf \emph{\underline{Instead:}}} Describe your experiences without using other people's full names.
    }
    \item[]{\bf \emph{If you send us anything that breaks these guidelines, we will not publish it, and may remove your blog.}}

\end{itemize}



\section*{Can I get my photos, drawings, or blog posts mailed back to me?}
If you'd like us to return any post or photo to you, please include a return envelope and adequate postage to do so, and include a note explaining what you'd like us to send back.  In general, we can't send your letters back to you otherwise.  We don't hold on to hard copies for very long, so be sure to request returned materials at the time you send them in.



\section*{What if I want to remove something I previously sent?}
Send us a letter, and say clearly at the top that you want us to remove some blog posts.  Tell us the approximate dates and a brief description of the posts that you want removed, and we will remove them as soon as we can.



\section*{How can I increase my chances to get connected to other people?}
\begin{itemize}
    \item{ Be brief.\\ \emph{Visitors are unlikely to stick through a letter longer than 5 or 6 pages; it's better to be closer to 1 or 2 pages.}}
    \item{ Be expressive.\\ \emph{Talk about your experience. Illustrate it.  You have lived a unique life, and your experiences are compelling and powerful.  The most popular stories tell people on the outside about things that are new to them.}}
    \item{ Write often -- but not too often.\\ \emph{One or two posts per week is a good rate.  Too much more than that, and readers may become overwhelmed and lose interest.}}
    \item{ When visitors leave comments, respond to them! }
\end{itemize}



\section*{When you write a blog post, please consider the following:}
\begin{itemize}
    \item{ Write as legibly as possible, using dark ink or type on clean paper.  This helps the scans come out much more clearly.  Fancy handwriting is great, just make sure it's easy to read.}
    \item{ Don't staple pages together -- we just have to remove the staples, and the holes in the paper can jam up the scanner.}
    \item{ Avoid folding too many times.  Normal tri-fold envelope folding is fine, but if the pages get too crinkled, it can be harder to scan. }
\end{itemize}



\section*{Other questions?}
If you have a question we didn't answer here (or a suggestion), please don't hesitate to write us.

\begin{itemize}
    \item[] {{ letter.org.mailing_address|tex_escape }}
\end{itemize}


\end{document}
