

Greetings and welcome,

We have received your signed consent form and have created a blog for you on
our web site. You can tell anybody with access to the Internet how to visit
your blog by giving them the following ``URL'' or Internet address:

\begin{quote}
\url{ {{ letter.recipient.profile.get_blog_url|absolute_url|tex_escape }} }
\end{quote}

As we explained in our introductory materials, we will publish all
material that you send to us.

There is no single style or form for blog posts and every blog is
different. That said, blog posts should be things you have written
yourself.  Most blog posts tend to be between 1 and 5 pages long but
can be longer or very short. Authors of blogs usually write about
their own thoughts or experiences but some use their blogs to share
artwork, poetry, fiction, public ``diary'' entries. Others share open
letters to world about issues they care about. Keeping in mind the
very public nature of the venue, you should feel free to send us
whatever you wish to share.

To show you how the post appears on the web, we will send you a
printout of your first post.  However, we will \emph{not} send you
written confirmation each time we receive and post material you send
us. If we cannot publish material you send for any reason, or if we
must remove material you have published, we will always contact you to
tell you. Unless we contact you to tell you otherwise, we have
published your material on the Internet.

Visitors who read your blog will have the opportunity to leave public
``comments.'' These comments will be visible to any other visitor on
the website. Periodically and as frequently as possible -- probably
every 1-2 weeks -- we will mail you any new comments that visitors
leave on material you have published. If no new comments are left, we
will not contact you to tell you.  Keep in mind that most viewers will
not leave comments and a lack of comments does not mean a lack of
people reading the posts you have shared.

We cannot provide you with any means to contact or ``email'' people
leaving comments either privately or directly. Of course, you are
welcome to send us new material that responds to comments you
receive. Commenters may, or may not, return to your blog to view your
responses.

Please remember that these exchanges, like all information published
using our service, will be completely public and listed under your
full name.

 {{ letter.org.personal_contact|tex_escape }} 
